% Please do not change the document class
\documentclass{scrartcl}

% Please do not change these packages
\usepackage[hidelinks]{hyperref}
\usepackage[none]{hyphenat}
\usepackage{setspace}
\doublespace

% You may add additional packages here
\usepackage{amsmath}

% Please include a clear, concise, and descriptive title
\title{Interfaces and Interactions Research Journal}

% Please do not change the subtitle
\subtitle{COMP210 - Research Journal}

% Please put your student number in the author field
\author{1605240}

\begin{document}

\maketitle

\section{Hardware Interfaces for VR Applications: Evaluation on Prototypes \cite{mentzelopoulos2015hardware}}
Paper begins by introducing VR and telling us how important it's for the player to feel immersed. The feeling of being immersed and instantly knowing what to do makes for a much realer experience. The paper then evaluates two different VR games with different inputs, one an xbox controller, the other a hyrda controller. These were tested on students who had a gaming background. The students were initially more comfortable using the xbox controller as that is what most people are used to, however after time and guidance they were equally as easy to use. Should be used to hardware before playing to get the best experience. Students found the game with the hydra controller was more enjoyable and had a greater sense of immersion compared to the game with xbox controller, it gave some of them motion sickness. Motion sickness is big part of VR, tinkering and research is needed to reduce this, they need to find a good solution for walking in VR. Paper predicts the future technologies will make VR great on a big scale, I agree as there is so much to do in VR.

\section{Exploitation of heuristics for virtual environments \cite{hvannberg2012exploitation}}
Nielsen's heuristics have been the normal heuristics for a long time however patchworks of heuristics might be more useful than a single set list, I agree because you can merge patches that your software fits into and pick the heuristics from there and get ones which are relevant which will save time. The paper looks at the idea of heuristics proposed by Sutcliffe and Gault. They found that most papers only used selected heuristics but most people still adapted them for their needs. This paper was useful to us students because it gives us an idea of how to do our own heuristics tests and shows how they can help uncover all the problems. Two of the twelve heuristic categories didn't make a feature in VR which is why they suggested modifying current heuristics.

\section{Investigating the balance between virtuality and reality in mobile mixed reality UI design: user perception of an augmented city \cite{venta2014investigating} }
This paper discusses peoples perceptions of virtual and real world representations of mobile UI's. To see how peoples perceptions varied between mobile mixed reality and VR, they made two phone applications that simulated mobile mixed reality, one was an augmented camera view, the other was a virtual 3D city model. The author wants this paper to help him understand how to fuse the two together, this is useful to anyone wanting to make an AR/VR game. There were strengths and weaknesses' about both. The 3D model was easier to recognize, however survey showed most users preferred the UI of the camera mode, it gave them more realism. Negative effect for camera is battery life and bad ergonomics and for 3D model they were slow processor on phone and hard to associate with reality. It recommends letting user pick between the two depending on current situation and to make hybrid versions of the UI. Both have drawbacks but they've found potential ways to try and solve this issue based on study results which is good as they have real feedback so this can help us make our own UI.

\section{Homuncular Flexibility in Virtual Reality.\cite{won2015homuncular} }
This paper talks about how VR allows user to inhabit avatar bodies different to their own, thus can produce side effects that are both psychological and physiological so if they avatar they are using have different limbs they start to act like they can control them. This is a very interesting topic. An experiment was done to show that users gained a sense of ownership of a rubber hand when it was placed in front of them on the table and stroked, they even flinched when they felt it was threatened. This is the feeling they want to create so you feel more immersed in the game, they want to trick your brain. This method will be extremely useful when creating immersion howeve has the potential to be taken too far eg in horror games. Participants were able to adapt to a new avatar in less than 10 minutes. To get this method to work smoothly in VR the movements have to be tracked precisely, it will take more work before this can be perfected. This paper is telling us that we can make wide ranges of avatars in VR and with enough time be able to adapt to how they move and feel ownership.

\section{(Re-)examination of multimodal augmented reality \cite{rosa2016re} }
There isn't a solid definition of what AR really is due to peoples perceptions. Milgram and Kishino decided there should be concrete definitions of what counts as different types of reality. They decide to measure real and virtual objects in three layers. The first layer is when a real objects exists and a virtual objects exists but isn't there in reality. The second layer is when you can view real world objects regardless but virtual objects can only be viewed indirectly. The third layer to this is made between real and virtual images, real life images have the right luminosity whereas with a virtual image they have none or they are transparent. Talks about mediated stimulus in AR and how it is not possible to change the physical objects but you can modify it to work in AR, EG looking at your arm in AR and the skin changing colour or hearing someones voice as a robot. Can use various forms of sensory substitution like in Luigi's Ghost Mansion where the player feels vibrations to sense if ghosts, this method can be put to good use in games. Split into real, meditated and virtual. Real is the physical world without being mediated. Meditated originates from real world but is perceived through digital forms. Virtual is computer generated objects viewed through a digital form. Overall this paper tries to discuss the different ways people view AR, although I found this paper boring and didn't take much away from reading it.

\section{Real Virtuality: A Code of Ethical Conduct. Recommendations for Good Scientific Practice and the Consumers of VR-Technology \cite{madary2016real} }
This paper discusses a list of ethical concerns regarding VR and such technology and recommends how to minimize said risks. Talks about plasticity of the human mind, says that our environment can influence our psychological state and we won't even know about it. Talks about rubber hand as mentioned above. Experiment was done where normal people played prison guards and prisoners, they began to show their pathological behavioral traits. The writer discusses VR and how it can have negative real world effects on the users health especially their mental health as everything is getting more life like, users might not be able to distinguish the difference. Not enough long term studies have been done to see if the use of VR over a prolonged period can have negative effects on your health as of yet. I believe over a long period using VR for hours a day will lead to a negative effects on peoples mental health. This paper is helpful as it discusses the future for VR and what to think about as they become more widely available.

\section{Conclusion}
First paper is helping us understand VR development in detail. Second paper is helping us understand heuristics tests and how helpful they can be when used correctly. Third paper is helping us think about interfaces for our AR/VR project and how we can make the best ones. The fourth paper makes us think about the differences between the types of reality. The last paper discusses future of VR and health effects it may cause. Overall these papers are helping us to understand VR in depth and different aspects of it. They are helpful to us as we should take the advice they give when making our game for this module.

\bibliographystyle{IEEEtran}
\bibliography{references}

\end{document}
